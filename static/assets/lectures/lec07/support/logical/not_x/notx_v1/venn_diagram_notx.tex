\documentclass[convert={density=300,size=1080x800,outext=.png}]{standalone}

\usepackage{tikz}
\begin{document}
      \def\firstduocircle{(0,0) circle (1.5cm)}%
      \def\secondduocircle{(0:2cm) circle (1.5cm)}%
      \colorlet{circle edge}{black}%
      \colorlet{circle area}{cyan!50}%
      \colorlet{circle area ans}{red!50}%
      \tikzset{filled/.style={fill=circle area, draw=circle edge, thick},
          ans/.style={fill=circle area ans, draw=circle edge, thick},
          outline/.style={draw=circle edge, thick}, line/.style={draw=circle edge, thick}}%
   
      \begin{tikzpicture}
          
		  % Color the background / sample space blue
		  \begin{scope}
              \draw[filled] (-1.75,-2.50) rectangle (3.75,2.45) node [below left] {$S$};
              \fill[white] \firstduocircle node [below, left] {$X$}
                                       \secondduocircle node [below, right] {$Y$};
          \end{scope}

		  % Color only Y
	      \begin{scope}
	          \clip \secondduocircle;
	          \draw[filled, even odd rule] \firstduocircle
	                                       \secondduocircle node {$Y$};
	      \end{scope}
          \draw[outline] \firstduocircle node {$X$}
                         \secondduocircle;
          \draw[thick] (-1.75,-2.50) rectangle (3.75,2.45) node [below left] {$S$};
          \draw[line] (-1.75,-3.25) -- (3.75,-3.25);
          \node[text=red] at (1,-3) {$P(X^c)$ is the same as $!X$};
      \end{tikzpicture}
	 
\end{document}
